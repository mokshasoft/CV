\item[2016-] \textbf{Sabbatical, Mokshasoft}

\textit{\textbf{Following my inspiration :)}}

There is something really important about taking time off now and then from whatever we are doing. The momentum (or entrapment) of past actions and goals become stronger and stronger the further we walk along the same path. Am I doing what I really want to, or am I choosing out of habit? Stefan Sagmeister touches upon it in his TED talk. So what I am currently doing is to slowly let the momentum come to a halt. I let intuition, interest and coincidences guide me. "What do I feel like doing today?"

So in practical terms, what does all that really mean and what have I done so far? I have:

\begin{itemize}
    \item visited a lot of friends
    \item done a lot of hiking, both with friends and on my own
    \item taken an abstract algebra course at Lund University
    \item been part of starting an eco-village, sunnemo.org
    \item learnt about permaculture and have literally gotten my hands dirty
    \item ported Idris to the formally verified OS seL4
    \item improved rebuild times 300x on TI StarterWare bare-metal programming using CMake
\end{itemize}

\textit{keywords: seL4, Haskell, Idris, 3D modelling, abstract algebra}
