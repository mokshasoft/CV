\item[2012-2014] \textbf{Software Security, Sony Mobile}

\textit{\textbf{Senior Software Consultant, ÅF/Epsilon}}

A manager at the DRM Department heard that I was leaving the multimedia department and really wanted me to stay in the company, so basically he offered me a position at either the DRM Department or the SW Security Department. I went on the interview, was offered the position and accepted on the same day I quit my previous assignment.

The SW Security department was in the beginning mostly responsible for key management of DRM keys. The component that handled it had been growing for many years without many architectural considerations, and was getting increasingly difficult to integrate on new platforms. My first achievement was to introduce clear integration APIs for this component with one piece in Linux user-space and one piece in ARM TrustZone. Knowledge of shell and Coccinelle scripting shortened this task considerably.

The responsibility of the SW Security department grew over time, and I worked on many system features, mostly in Linux user-space and kernel-space. Leading the low-level integration of file-level encryption using eCryptfs was a really worthwhile task. The widened responsibility made it possible for me to learn a lot about Linux in general, to learn profiling on Linux using oprofile, and to learn about the SELinux integration on android called SEAndroid.

\textit{keywords: Linux, Android, SELinux, crypto, TrustZone, eCryptfs, oprofile, enterprise}